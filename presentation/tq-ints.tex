\begin{frame}{Twenty Questions}
  \begin{block}{Refinement by Example}
    Attempt to ask interesting questions about candidates

    \begin{itemize}
      \item spec can be many things, e.g. types, constraints
      \item results ranked by ``interestingness''
      \item refinement by questions
      \item refinement by selection
      \item refinement by spec
    \end{itemize}

    TODO diagram float bottom right
  \end{block}
\end{frame}


\begin{frame}{Twenty Questions}
  \begin{block}{Is it alive?}
    Provide a spec, \textit{automatically} differentiate

    \begin{example}
      \lstinputlisting[language=bash]{examples/tq-1}
    \end{example}
  \end{block}
\end{frame}

\begin{frame}{Twenty Questions}
  \begin{block}{Is it a person?}
    Decisions extend the specification

    \begin{example}
      \lstinputlisting[language=bash]{examples/tq-2}
    \end{example}
  \end{block}
\end{frame}

\begin{frame}{Twenty Questions}
  TODO take this further and use traces to present ``interesting inputs''
\end{frame}


\begin{frame}{Twenty Questions}
  \begin{block}{Noteworthy Details}
    There are limitations here too

    \begin{itemize}
      \item exponentially small \textit{interesting} inputs
      \item side effects mostly ignored
      \item random inputs confusing at best
    \end{itemize}
  \end{block}
\end{frame}

\begin{frame}{Twenty Questions}
  TODO should we walk through our thinking at this point? Discuss how we arrived at the conclusion that trace information is crucial for input generation and further expression candidate refinement

  would make for a nice segue into the oauth example that uses traces directly
\end{frame}
