\documentclass{beamer}
\usefonttheme{serif}
\usepackage{semantic}
\usepackage{palatino}
\usepackage{listings}
\usepackage{mathtools}
\usepackage{xstring}
\usepackage{syntax}
\usepackage{xcolor}
\usepackage{tikz}

\renewcommand{\syntleft}{\normalfont\itshape}
\renewcommand{\syntright}{}

\newcommand{\highlight}[1]{\colorbox{yellow}{$\displaystyle #1$}}

\newcommand{\cmp}[2]{
  \mathcal{#1}\,|[ \mathtt{ \StrSubstitute{#2}{ }{\ } } |]
}

\lstdefinelanguage{diff}{
  morecomment=[f][\color{blue}]{@@},
  morecomment=[f][\color{red}]-,
  morecomment=[f][\color{green}]+,
  morecomment=[f][\color{magenta}]{---},
  morecomment=[f][\color{magenta}]{+++},
}

\definecolor{solarized@base03}{HTML}{002B36}
\definecolor{solarized@base02}{HTML}{073642}
\definecolor{solarized@base01}{HTML}{586e75}
\definecolor{solarized@base00}{HTML}{657b83}
\definecolor{solarized@base0}{HTML}{839496}
\definecolor{solarized@base1}{HTML}{93a1a1}
\definecolor{solarized@base2}{HTML}{EEE8D5}
\definecolor{solarized@base3}{HTML}{FDF6E3}
\definecolor{solarized@yellow}{HTML}{B58900}
\definecolor{solarized@orange}{HTML}{CB4B16}
\definecolor{solarized@red}{HTML}{DC322F}
\definecolor{solarized@magenta}{HTML}{D33682}
\definecolor{solarized@violet}{HTML}{6C71C4}
\definecolor{solarized@blue}{HTML}{268BD2}
\definecolor{solarized@cyan}{HTML}{2AA198}
\definecolor{solarized@green}{HTML}{859900}

\lstset{
  upquote=true,
  columns=fixed,
  tabsize=2,
  extendedchars=true,
  breaklines=true,
  numbersep=5pt,
  rulesepcolor=\color{solarized@base03},
  numberstyle=\tiny\color{solarized@base01},
  basicstyle=\ttfamily\footnotesize,
  keywordstyle=\color{solarized@green},
  stringstyle=\color{solarized@cyan}\ttfamily,
  identifierstyle=\color{solarized@blue},
  commentstyle=\color{solarized@base01},
  emphstyle=\color{solarized@red},
  framextopmargin=0pt,
  frame=lines,
  rulecolor=\color{solarized@base2}
}

% There are many different themes available for Beamer. A comprehensive
% list with examples is given here:
% http://deic.uab.es/~iblanes/beamer_gallery/index_by_theme.html
% You can uncomment the themes below if you would like to use a different
% one:
%\usetheme{AnnArbor}
%\usetheme{Antibes}
%\usetheme{Bergen}
%\usetheme{Berkeley}
%\usetheme{Berlin}
%\usetheme{Boadilla}
\usetheme{boxes}
%\usetheme{CambridgeUS}
%\usetheme{Copenhagen}
%\usetheme{Darmstadt}
%\usetheme{default}
%\usetheme{Frankfurt}
%\usetheme{Goettingen}
%\usetheme{Hannover}
%\usetheme{Ilmenau}
%\usetheme{JuanLesPins}
%\usetheme{Luebeck}
%\usetheme{Madrid}
%\usetheme{Malmoe}
%\usetheme{Marburg}
%\usetheme{Montpellier}
%\usetheme{PaloAlto}
%\usetheme{Pittsburgh}
%\usetheme{Rochester}
%\usetheme{Singapore}
%\usetheme{Szeged}
%\usetheme{Warsaw}

%remove navigation symbols
\setbeamertemplate{navigation symbols}{}

\title{Spec by Example}

% A subtitle is optional and this may be deleted
\subtitle{Interactive Specification Refinement}

\author{Matt Brown, John Bender}
% - Give the names in the same order as the appear in the paper.
% - Use the \inst{?} command only if the authors have different
%   affiliation.

\date{CS 239, Spring 2014}
% - Either use conference name or its abbreviation.
% - Not really informative to the audience, more for people (including
%   yourself) who are reading the slides online

\subject{Programming Languages and Systems, Computer Science}
% This is only inserted into the PDF information catalog. Can be left
% out.

% If you have a file called "university-logo-filename.xxx", where xxx
% is a graphic format that can be processed by latex or pdflatex,
% resp., then you can add a logo as follows:

% \pgfdeclareimage[height=0.5cm]{university-logo}{university-logo-filename}
% \logo{\pgfuseimage{university-logo}}

% Delete this, if you do not want the table of contents to pop up at
% the beginning of each subsection:
\AtBeginSubsection[]
{
  \begin{frame}<beamer>{Outline}
    \tableofcontents[currentsection,currentsubsection]
  \end{frame}
}


% Let's get started
\begin{document}

\setlength{\abovedisplayskip}{0pt}
\setlength{\belowdisplayskip}{0pt}
\setlength{\abovedisplayshortskip}{0pt}
\setlength{\belowdisplayshortskip}{0pt}

\begin{frame}
  \titlepage
\end{frame}

\begin{frame}{Interactive Specification Refinement}
  \begin{block}{Spec by Example by Example}
    Writing specifications is hard, synthesis is inaccurate.

    \begin{enumerate}
      \item define synthesis
      \item existing work
      \item areas for improvement
      \item advanced synthesis
    \end{enumerate}
  \end{block}
\end{frame}

\begin{frame}{Automatic Program Construction}
  \begin{block}{Have the Computer Build It!}
    define synthesis
  \end{block}
\end{frame}

\begin{frame}{Automatic Program Construction}
  \begin{block}{Synthesizzle or Synthesnooze}
    Synthesis is very cool, but currently limited

    \begin{itemize}
      \item automatic selection inaccurate
      \item manual selection tedious
      \item only works for small domains
    \end{itemize}
  \end{block}
\end{frame}

\begin{frame}{Jungloid}
  \begin{block}{Bienvenido al Jungloid}
    Type based synthesis for Java expressions

    \begin{itemize}
      \item ``spec'', e.g. $FactoryFactory \rightarrow BeanFactory$
      \item results ranked by size
      \item no tools for refinement
      \item simple test cases, e.g. $Map \rightarrow Iterator$
    \end{itemize}
  \end{block}
\end{frame}

\begin{frame}{Jungloid}
  \begin{block}{From Integers to an Integer}
    No help past candidate generation and ranking

    \begin{example}
      \lstinputlisting[language=bash]{examples/jungloid}
    \end{example}
  \end{block}
\end{frame}


\begin{frame}{CodeHint}
  \begin{block}{Manual Candidate Set Refinement}
    Like Jungloid, requires debugger, manual refinement

    \begin{itemize}
      \item ``spec'', e.g. $BeanFactoryFactory$ + debugging context
      \item results ranked by $|$expressions$|$
      \item refinement by further execution
      \item refinement by common methods, e.g. $toString$
    \end{itemize}

    TODO diagram float bottom right
  \end{block}
\end{frame}

\begin{frame}{CodeHint}
  \begin{block}{Break and Burn}
    Stop at a breakpoint, find an expression

    \begin{example}
      \lstinputlisting[language=bash]{examples/codehint-1}
    \end{example}
  \end{block}
\end{frame}

\begin{frame}{CodeHint}
  \begin{block}{Refine by Execution}
    Continue by executing each of the expressions

    \begin{example}
      \lstinputlisting[language=bash]{examples/codehint-2}
    \end{example}
  \end{block}
\end{frame}

\begin{frame}{CodeHint}
  \begin{block}{Refine by Comparison}
    Continue by executing each of the expressions

    \begin{example}
      \lstinputlisting[language=bash]{examples/codehint-3}
    \end{example}
  \end{block}
\end{frame}

\begin{frame}{CodeHint}
  \begin{block}{Noteworthy Details}
    CodeHint has limitations outside its manual nature

    \begin{itemize}
      \item debugging context is restrictive
      \item side effects impossible to account for
      \item candidate differentiation is manual
      \item full knowledge of desired outcome required
    \end{itemize}
  \end{block}
\end{frame}


\begin{frame}{Twenty Questions}
  \begin{block}{A Better Approach}

    \begin{itemize}
      \item Why are there 50 candidates?
      \item How are these candidates different?
      \item Highlight the differences by example
    \end{itemize}
  \end{block}
\end{frame}


\begin{frame}{Twenty Questions}
  \begin{block}{Is it alive?}
    Provide a spec, \textit{automatically} differentiate

    \begin{example}
      \lstinputlisting[language=bash]{examples/tq-1}
    \end{example}
  \end{block}
\end{frame}

\begin{frame}{Twenty Questions}
  \begin{block}{Is it a person?}
    Decisions extend the specification

    \begin{example}
      \lstinputlisting[language=bash]{examples/tq-2}
    \end{example}
  \end{block}
\end{frame}

\begin{frame}{Twenty Questions}
  \begin{block}{How do we do this?}
    There are still issues with this approach

    \begin{itemize}
      \item Finding examples
        \begin{itemize}
          \item Random
          \item Systematic with trace
        \end{itemize}
      \item Describing the behaviors to the user
        \begin{itemize}
          \item Output
          \item Trace
          \item Side-effects
        \end{itemize}
    \end{itemize}
  \end{block}
\end{frame}

\newcommand{\lsthaskell}[1]{\lstinline[language=haskell]{#1}}
\newcommand{\Q}{{\color{red}Q:}\hspace{2mm}}
\newcommand{\A}{{\color{blue}A:}\hspace{2mm}}

\begin{frame}{A parser in 22 questions}
  Define a parser for arithmetic expressions by example.

\[
  t ::= \langle number \rangle ~|~ ( t ) ~|~ t + t ~|~ t - t ~|~ t * t ~|~ t / t
\]

{\bf Goal:} avoid the usual problems with parsing:\\
\begin{itemize}
  \item{Associativity}
  \item Precedence
  \item Left-recursion
  \item Ambiguity
  \item $\dots$
\end{itemize}
\end{frame}

\begin{frame}[fragile]
\frametitle{First Steps}

  Start by defining the Abstract Syntax type (mirrors the BNF):

\begin{center}
\begin{minipage}{.7\textwidth}
$t ::= \langle number \rangle ~|~ ( t ) ~|~ t + t ~|~ t - t ~|~ t * t ~|~ t / t$

\begin{lstlisting}[mathescape,language=haskell]
data T = Num Int | Paren T
       | Add T T | Sub T T
       | Mul T T | Div T T

parseT :: String $\to$ T
parseT = ???
\end{lstlisting}
\end{minipage}
\end{center}

\begin{itemize}
\item{The type T defines the possible cases}
\item{Use to check if a specification is complete}
\item{Want to synthesize $parseT$}
\end{itemize}
\end{frame}

\begin{frame}[fragile]
\frametitle{Bootstrapping Examples}
  Interactive Q/A session (22 questions in this case):

\begin{itemize}
  \item A ``bootstrapping'' example for each case (6 examples)
  \item {Detect and resolve ambiguities:
    \begin{itemize}
      \item Associativity of each operator (4 examples)
      \item Precedence rules (12 examples)
    \end{itemize}
  }
\end{itemize}

\end{frame}

\begin{frame}[fragile]
\frametitle{Bootstrapping Examples}

\begin{itemize}
\item Collect a simple example for each case
\item Generalize to produce an initial specification
\end{itemize}

\vspace{5mm}

\Q Does ``123'' parse as \lsthaskell{Num 123}? (Y/N)\\
\A Y
\vspace{2mm}

\Q Which input parses as \lsthaskell{Paren (Num 0)}?\\
\A ``(0)''\\
\vspace{2mm}

\Q Which input parses as \lsthaskell{Add (Num 1) (Num 2)}?\\
\A ``1+2''\\
\vspace{2mm}

\Q Which input parses as \lsthaskell{Sub (Num 1) (Num 2)}?\\
\A ``1-2''\\

$\dots$

\end{frame}

\begin{frame}[fragile]
\frametitle{Detecting Ambiguity}

The specification can be used to synthesize a {\em pretty-printer},
which {\em must} be the right inverse of a parser:

\begin{lstlisting}[mathescape, language=haskell]
printT :: T -> String
$\forall$t:T. parseT (printT t) == t
\end{lstlisting}

\begin{itemize}
\item{\lsthaskell{printT} must be one-to-one}
\item{If \lsthaskell{printT} maps two distinct ASTs to the same
    string, our specification is ambiguous}
\item{Automatically detect ambiguities}
\item{Ask questions to resolve ambiguities}
\end{itemize}
\end{frame}

\begin{frame}[fragile]
\frametitle{Resolving Ambiguities}

\begin{itemize}
\item Ask user to choose between possible parses of a string
\end{itemize}

\vspace{5mm}

\Q What is the parse of ``1+2*3''? \hspace{2mm} (A/B)

\hspace{6mm}\begin{minipage}{.4\textwidth}
{\bf Choice 1:}\\
\begin{tikzpicture}
  \node(root) at (3,3){Mul};
  \node(l) at (2,2){Add};
  \node(ll) at (1,1){Num 1};
  \node(lr) at (3,1){Num 2};
  \node(r) at (4,2){Num 3};

  \draw (root) -- (l);
  \draw (root) -- (r);
  \draw (l) -- (ll);
  \draw (l) -- (lr);
\end{tikzpicture}
\end{minipage}
\hspace{5mm}
\begin{minipage}{.4\textwidth}
{\bf Choice 2:}\\
\begin{tikzpicture}
  \node(root) at (3,3){Add};
  \node(l) at (2,2){Num 1};
  \node(r) at (4,2){Mul};
  \node(rl) at (3,1){Num 2};
  \node(rr) at (5,1){Num 3};

  \draw (root) -- (l);
  \draw (root) -- (r);
  \draw (r) -- (rl);
  \draw (r) -- (rr);
\end{tikzpicture}
\end{minipage}
\\
\A 2
\vspace{2mm}

\begin{itemize}
\item{Choice 1 flagged as an invalid AST (can't be printed)}
\item{Similar questions resolve all associativity and precedence
    ambiguities}
\end{itemize}
\end{frame}

\begin{frame}[fragile]
\frametitle{The complete {\small(abridged)} specification}

\begin{itemize}
\item{In 22 examples}
\end{itemize}

\begin{lstlisting}[mathescape,language=haskell]
-- bootstrapping
parse "123" == Num 123
parse "(0)" == Paren (Num 0)
parse "1+2" == Add (Num 1) (Num 2)   (x4)

-- associativity
parse "1+2+3" ==
  Add (Add (Num 1) (Num 2)) (Num 3)  (x4)

-- precedence
parse "1+2*3" ==
  Add (Num 1) (Mul (Num 2) (Num 3))  (x12)
\end{lstlisting}

\end{frame}

\begin{frame}{Takeaways}
\begin{itemize}
\item Synthesize parser and pretty printer simultaneously
\item Bootstrap to cover all cases
\item Detect and resolve ambiguities
\item Use heuristics to simplify questions and extrapolate examples
\end{itemize}
\end{frame}

\begin{frame}{Questions?}

\end{frame}

\end{document}
